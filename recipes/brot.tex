\begin{recipe}
				[ % 
								preparationtime = {\unit[15]{Minuten}},
								bakingtime = {\unit[1]{Stunde}},
								portion = {\portion{2}},
								calory,
								source = Internet
				]
				{Brot}

				\graph
				{% pictures
								small,
								big=photos/brot-big
				}

				\ingredients
				{%
								\unit[250]{g} 	& Mehl\\
								\unit[1]{Prise}	& Salz\\
								\unit[150]{ml}	& lauwarmes Wasser\\
								\unit[\nicefrac{1}{2}]{Pack}	& Trockenhefe\\
								\unit[1]{TL} & Zucker\\
								\\
								\multicolumn{2}{c}{\textbf{Utensilien}} \\
								1 & großen Topf (\unit[1.75]{l})\\ 
								1 & Trangiapfanne\\
								6 & Teelichter
				}

				\preparation
				{%
								\step \unit[250]{g} Mehl und \unit[1]{Prise} Salz auf einen Teller geben.
								\step \unit[\nicefrac{1}{2}]{Pack} Trockenhefe und \unit[1]{TL} Zucker in \unit[150]{ml} lauwarmes Wasser geben und warten, bis die Hefe aktiv wird (Bläschenbildung).
								\step In kleinen Schritten die Hefe zu dem Mehl geben und verkneten.
								\step Großen Trangiatopf mit Backpapier auskleiden.
								\step Teig in den Topf geben und leicht flachdrücken. Pfanne als Deckel verwenden und 10 bis 15 Minuten gehen lassen.
								\step \unit[6]{Teelichter} im Kreis auf den Mittelboden des Sturmkochers stellen und anzünden. Den Topf darauf stellen.
								\step Nach \unit[35]{Minuten} das Brot wenden.
								\step Nach weiteren \unit[25]{Minuten} das Brot aus dem Topf holen.
				}

				\hint
				{%
								Damit der Hefeteig auch im Freien bei etwas niedrigeren Temperaturen geht, einfach mittig unter den Sturmkocher ein brennendes Teelicht stellen und den Teig im Topf über dem Teelicht gehen lassen.
				}

\end{recipe}


