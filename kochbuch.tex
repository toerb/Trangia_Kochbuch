\documentclass[%
a4paper,
%twoside,
11pt
]{article}

% encoding, font, language
\usepackage[T1]{fontenc}
\usepackage[latin1]{inputenc}
\usepackage{lmodern}
\usepackage[ngerman, english]{babel}

\usepackage{nicefrac}

\usepackage[
    %handwritten,
    nowarnings,
    %myconfig
]
{xcookybooky}

\definecolor{mygreen}{rgb}{0,.5,0}
\DeclareRobustCommand{\textcelcius}{\ensuremath{^{\circ}\mathrm{C}}}

\setRecipeColors
{%
    recipename = mygreen,
    ing = blue,
    inghead = blue,
    prep,
    prephead,
    hint,
    hinthead,
}

\setcounter{secnumdepth}{1}
\renewcommand*{\recipesection}[2][]
{%
     \subsection[#1]{#2}
}
\renewcommand{\subsectionmark}[1]
{% no implementation to display the section name instead
}


%%%%%%%%%%
% hyperref
\usepackage{hyperref}    % must be the last package
\hypersetup{%
    pdfauthor            = {Tobias Gro�},
    pdftitle             = {Trangia Kochbuch},
    pdfsubject           = {Kochbuch},
    pdfkeywords          = {Kochbuch},
    pdfstartview         = {FitV},             
    pdfview              = {FitH},
    pdfpagemode          = {UseNone}, % Options; UseNone, UseOutlines
    bookmarksopen        = {true},
    pdfpagetransition    = {Glitter},
    colorlinks           = {true},
    linkcolor            = {black}, 
    urlcolor             = {black}
    citecolor            = {black}, 
    filecolor            = {black},
}
% hyperref
%%%%%%%%%%



\begin{document}

\title{Kochbuch f�r Trangiakocher}
\author{Tobias Gro�}
\maketitle

\tableofcontents

\vspace{9em}

\section{Recipes}

\newpage

% background graphic
%\setBackgroundPicture[x, y=-2cm, width=\paperwidth-4cm, height, orientation = pagecenter]
%{pic/background}

\begin{recipe}
[ % 
    preparationtime = {\unit[15]{Minuten}},
		bakingtime = {\unit[1]{Stunde}},
    bakingtemperature,
    portion = {\portion{2}},
    calory,
    source = Internet
]
{Brot}
    
    \graph
		{% pictures
		small,
		big
		}
																		
		\ingredients
    {%
				\unit[250]{g} 	& Mehl\\
				\unit[1]{Prise}	& Salz\\
				\unit[150]{ml}	& lauwarmes Wasser\\
				\unit[$\frac{1}{2}$]{Pack}	& Trockenhefe\\
        \unit[1]{TL} & Zucker\\
				\\
				\multicolumn{2}{l}{\textbf{Utensilien}} \\
				\\
				1 & gro�en Topf (\unit[1.75]{l})\\ 
				1 & Trangiapfanne\\
				6 & Teelichter
    }
    
    \preparation
    {%
				\step \unit[250]{g} Mehl und \unit[1]{Prise} Salz auf einen Teller geben.
				\step \unit[$\frac{1}{2}$]{Pack} Trockenhefe und \unit[1]{TL} Zucker in \unit[150]{ml} lauwarmes Wasser geben und warten, bis die Hefe aktiv wird (Bl�schenbildung).
				\step In kleinen Schritten die Hefe zu dem Mehl geben und verkneten.
				\step Gro�en Trangiatopf mit Backpapier auskleiden.
				\step Teig in den Topf geben und leicht flachdr�cken. Pfanne als Deckel verwenden und 10 bis 15 Minuten gehen lassen.
				\step \unit[6]{Teelichter} im Kreis auf den Mittelboden des Sturmkochers stellen und anz�nden. Den Topf darauf stellen.
				\step Nach \unit[35]{Minuten} das Brot wenden.
				\step Nach weiteren \unit[25]{Minuten} das Brot aus dem Topf holen.
    }
    
    \hint
    {%
        Damit der Hefeteig auch im Freien bei etwas niedrigeren Temperaturen geht, einfach mittig unter den Sturmkocher ein brennendes Teelicht stellen und den Teig im Topf �ber dem Teelicht gehen lassen.
    }

\end{recipe}

\newpage % if you separate your recipes into single files this command is not necessary

\end{document} 
